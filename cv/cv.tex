\documentclass[10pt]{article}
%\usepackage[cmex10]{amsmath}
\usepackage{array}
\usepackage{mdwmath}
\usepackage{wrapfig}
%\usepackage{ae}
\usepackage[T1]{fontenc}
\usepackage{tgpagella}
\usepackage{float}
\usepackage{enumitem}
%\usepackage[latin1]{inputenc}
\usepackage{mdwtab}
\usepackage{amssymb}
\usepackage{eqparbox}
\usepackage{eulervm}
\usepackage{geometry}
\usepackage{subfig}
\usepackage{graphicx}
\usepackage{setspace}
\usepackage{url}
%\usepackage{lmodern}
\geometry{lmargin=2cm,tmargin=2cm,rmargin=2cm,bmargin=1cm}
%opening
\setlength\parindent{0pt}

\begin{document}
\pagestyle{empty}
\begin{titlepage}
     {\Large{\textbf{Jos\'e Vin\'icius de Miranda Cardoso}}}
     \vspace{.5cm}

    \begin{minipage}[b]{8cm}
      keywords: machine learning, time series, optimization, software development\\
    \end{minipage}
    \hfill
    \begin{minipage}[b]{4cm}
        \hfill \texttt{jvdmc@connect.ust.hk}\\
        \texttt{https://mirca.github.io}\\
        \texttt{GitHub: @mirca}
    \end{minipage}


\section*{Education}

\emph{PhD Student in Electronic and Computer Engineering} \hfill \textit{Fall 2019 -- Current} \\
\textbf{The Hong Kong University of Science and Technology}, Hong Kong\\

\emph{Visiting Student -- Electrical Engineering and Computer Science} \hfill \textit{Fall 2014 -- Spring 2015} \\
\textbf{The Catholic University of America}, USA\\
\textbf{University of Maryland at College Park}, USA \\
Brazil Scientific Mobility Program, Fully funded scholarship recipient \\

\emph{B.Eng. in Electrical Engineering} \hfill \textit{2019} \\
\textbf{Federal University of Campina Grande}, Brazil

\section*{Professional Experience}
\emph{Machine Learning Mentor} \hfill \textit{May 2019  -- Aug 2019}
\\\textbf{Udacity}, Remote
\vspace{.5cm}

\emph{Scientific Software Engineering Intern} \hfill \textit{Mar 2017 -- Feb 2018}
\\\textbf{NASA Ames Research Center}, Silicon Valley, CA, USA
\\Kepler/K2 Guest Observer Office
\vspace{.5cm}

\emph{Google Summer of Code Student} \hfill \textit{Summer 2016}
\\\textbf{The AstroPy Project}
\\ Project title: Point spread function photometry for fitting overlapping stars simultaneously
\vspace{.5cm}

\emph{Undergraduate Teaching Assistant} \hfill \textit{Spring 2015}
\\\emph{Probability and Statistics for Electrical Engineering and Computer Science}
\\\textbf{Federal University of Campina Grande}, Brazil
\vspace{.5cm}

\emph{Undergraduate Research Assistant} \hfill \textit{Fall 2015 -- Fall 2016}
\\\textbf{Institute for Advanced Studies in Communications}, Brazil
\vspace{.5cm}

\emph{Undergraduate Guest Researcher} \hfill \textit{Summer 2015}
\\\textbf{National Institute of Standards and Technology}, USA
\\Center for Nanoscale Science and Technology
\\Nanofabrication Research Group
\vspace{.5cm}


\section*{Volunteering Experience}
\emph{Deputy AstroPy GSoC Coordinator} \hfill \textit{Fall 2019 -- Current}
\\Deputy coordinator for the AstroPy project in the Google Summer of Code program
\vspace{.5cm}

\emph{Google Summer of Code Organization Administrator} \hfill \textit{Summer 2019 -- Current}
\\Admin for the OpenAstronomy organization during GSoC 2019
\vspace{.5cm}

\emph{Google Summer of Code Mentor for the AstroPy Project} \hfill \textit{Summer 2018}
\\Project title: Develop astropy tutorials on how to fit data
\vspace{.5cm}


\section*{Project Proposals}
    \textbf{NASA Transiting Exoplanet Survey Satellite Proposal}
    \hfill \textit{2019}\\
    Uniform Light Curves Across the Entire Sky from TESS FFIs with ELEANOR\\
    {\small Principal Investigators: Dr. Benjamin Montet (University of Chicago) and Dr. Jacob Bean (University of Chicago)}\\
    {\small Co-Investigators: Adina Feinstein (University of Chicago), Dr. Daniel Foreman-Mackey (Flatiron),
    Dr. Jessie Christiansen (IPAC/Caltech), Dr. Rodrigo Luger (U. of Washington),
    Dr. Daniel Scolnic (U. of Chicago), and Dr. Christina Hedges (NASA Ames),
  Nicholas Saunders (University of Hawaii), Jos\'e Vin\'icius de Miranda Cardoso (Universidade Federal de Campina Grande)}\\

    \textbf{NASA Transiting Exoplanet Survey Satellite Proposal}
    \hfill \textit{2018}\\
    Performing The Most Comprehensive Exoplanet Survey Of The Southern Sky With TESS Full Frame Images\\
    {\small Principal Investigator: Dr. Benjamin Montet (University of Chicago)}\\
    {\small Co-Investigators: Dr. Daniel Foreman-Mackey (Flatiron), Dr. Jessie Christiansen (IPAC/Caltech),
    Dr. Rodrigo Luger (U. of Washington), Dr. Daniel Scolnic (U. of Chicago), and Dr. Christina Hedges (NASA Ames)}\\
    {\small Undergraduate students: Nicholas Saunders (U. of Washington) and
      Jos\'e Vin\'icius de Miranda Cardoso (Universidade Federal de Campina Grande)}

\section*{Selected Publications}
\begin{enumerate}
  \item Kumar, S., Ying, J., \textbf{Cardoso, J. V. M.}, Palomar, D. P. Structured graph learning via Laplacian
  spectral constraints. Advances in Neural Information Processing Systems (NeurIPS), Dec. 2019.
  \item Kumar, S., Ying, J., \textbf{Cardoso, J. V. M.}, Palomar, D. P. A unified framework for structured graph
  learning via spectral constraints. Arxiv: \url{https://arxiv.org/pdf/1904.09792.pdf}, Apr. 2019.
\item Davanco, M., I., Liu, J., Sapienza, L., Zhang, C. Z., \textbf{Cardoso, J. V. M.}, Verma, V., Mirin, R., Nam,
S. W, Srinivasan, K. Heterogeneous integration for on-chip quantum photonic circuits with single quantum dot devices.
\textit{Nature Communications}, 2017.
\item \textbf{Cardoso, J. V. M.}, \textit{et. al.} An approximate exponentiated Weibull envelope-phase distribution.
\textit{IEEE International Symposium on Antennas and Propagation/USNC-URSI National Radio Science Meeting}, Farjado, Puerto Rico, 2016.
$\star\star$\textbf{\textit{Travel grant recipient}}$\star\star$.
\end{enumerate}

For a complete list of my publications, please refer to \url{https://mirca.github.io/publications}.

\section*{Awards}
\begin{enumerate}
  \item Selected, with full travel funding, to the workshop \textit{Preparing for TESS}, New York City, USA, 2018
  \item Selected to the workshop \textit{Python in Astronomy}, Leiden, The Netherlands, 2017
  \item Selected, with full travel funding, to the S\~ao Paulo School of Advanced Science on Nanophotonics, S\~ao Paulo, Brazil, 2016
  \item Travel Grant Recipient, IEEE Antennas and Propagation Symposium, Puerto Rico, 2016
  \item Young Author Recognition Award, International Telecommunication Union, ITU Kaleidoscope 2015
  \item Young Author Recognition Award, International Telecommunication Union, ITU Kaleidoscope 2014
  \item The paper ``SQUALES: A QT-based Application for Full-Reference Objective Stereoscopic
      Video Quality Measurement'' was one of the six papers nominated for Best Paper Award at ITU Kaleidoscope 2014
\end{enumerate}

\section*{Competencies}
\begin{description}
    \item[Coding:] Python (numpy, scipy, pandas, scikit-learn), R, git/GitHub, TensorFlow, C/C++, Unix shell, MATLAB
    \item[Courses:] Convex Optimization, Stochastic Processes, Information Theory, Random Signal Theory,
      Estimation and Detection Theory
    \item[Languages:] Native Portuguese, Fluent English
\end{description}

\end{titlepage}

\end{document}
