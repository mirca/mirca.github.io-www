\documentclass[10pt]{article}
%\usepackage[cmex10]{amsmath}
\usepackage{array}
\usepackage{mdwmath}
\usepackage{wrapfig}
%\usepackage{ae}
\usepackage[T1]{fontenc}
\usepackage{tgpagella}
\usepackage{float}
\usepackage{enumitem}
%\usepackage[latin1]{inputenc}
\usepackage{mdwtab}
\usepackage{amssymb}
\usepackage{eqparbox}
\usepackage{eulervm}
\usepackage{geometry}
\usepackage{subfig}
\usepackage{graphicx}
\usepackage{setspace}
\usepackage{url}
%\usepackage{lmodern}
\geometry{lmargin=2cm,tmargin=3cm,rmargin=2cm,bmargin=2cm}
%opening
\setlength\parindent{0pt}

\begin{document}
\pagestyle{empty}
\begin{titlepage}
     {\Large{\textbf{Jos\'e Vin\'icius de Miranda Cardoso}}}
     \vspace{.5cm}

    \begin{minipage}[b]{8cm}
     Undergraduate Student\\
     Federal University of Campina Grande, Brazil\\
     Department of Electrical Engineering\\
     Campina Grande, Brazil
    \end{minipage}
    \hfill
    \begin{minipage}[b]{4cm}
        \hfill \texttt{jvmirca@gmail.com}\\
        \texttt{http://mirca.github.io}
    \end{minipage}


\section*{Education}

\emph{Undergraduate in progress in Electrical Engineering} \hfill \textit{2011} --\\
\textbf{Federal University of Campina Grande}, Brazil\\
Advisor: Dr. Marcelo Sampaio de Alencar
\vspace{.5cm}

\emph{Visiting Student -- Electrical Engineering and Computer Science} \hfill \textit{Fall 2014 -- Spring 2015} \\
\textbf{The Catholic University of America}, USA\\
\textbf{University of Maryland at College Park}, USA \\
Brazil Scientific Mobility Program, Fully funded scholarship recipient \\
Advisors: Dr. Duilia F. de Mello and Dr. Jandro L. Abot
\vspace{.5cm}

\emph{Technical Degree in Informatics} \hfill \textit{2007 -- 2010}\\
\textbf{Federal Institute of Education, Science and Technology of Para\'iba}, Brazil \\
Advisor: Dr. Carlos Danilo Miranda Regis

\section*{Professional Experience}
\emph{Scientific Software Engineering Intern} \hfill \textit{Mar 2017 -- Feb 2018}
\\\textbf{NASA Ames Research Center}, Silicon Valley, USA
\\Kepler/K2 Guest Observer Office
\\Mentor: Dr. Geert Barentsen
\vspace{.5cm}

\emph{Software Developer at Google Summer of Code} \hfill \textit{Summer 2016}
\\\textbf{Google Summer of Code -- The AstroPy Project}
    \\Mentors: Dr. Erik Tollerud, Dr. Hans Moritz G\"unther, and Dr. Brigitta Sip\H{o}cz
\vspace{.5cm}

\emph{Undergraduate Teaching Assistant} \hfill \textit{Spring 2015}
\\\emph{Probability and Statistics for Electrical Engineering and Computer Science}
\\\textbf{Federal University of Campina Grande}, Brazil
\vspace{.5cm}

\emph{Undergraduate Research Assistant} \hfill \textit{Fall 2015 -- Fall 2016}
\\\textbf{Institute for Advanced Studies in Communications}, Brazil
\\Mentor: Dr. Marcelo Sampaio Alencar
\vspace{.5cm}

\emph{Undergraduate Guest Researcher} \hfill \textit{Summer 2015}
\\\textbf{National Institute of Standards and Technology}, USA
\\Center for Nanoscale Science and Technology
\\Nanofabrication Research Group
\\Mentor: Dr. Marcelo Ishihara Davan\c co
\vspace{.5cm}

\emph{Undergraduate Research Assistant} \hfill \textit{2011 -- 2014}
\\\textbf{Institute for Advanced Studies in Communications}, Brazil
\\Mentor: Dr. Marcelo Sampaio Alencar

\section*{Projects}
    \textbf{NASA Transiting Exoplanet Survey Satellite (TESS) Proposal}
    \hfill \textit{2018}\\
    Performing The Most Comprehensive Exoplanet Survey Of The Southern Sky With TESS Full Frame Images\\
    {\small Principal Investigator: Dr. Benjamin Montet (University of Chicago)}\\
    {\small Co-Investigators: Dr. Dan Foreman-Mackey (Flatiron), Dr. Jessie Christiansen (IPAC/Caltech),
    Dr. Rodrigo Luger (U. of Washington), Dr. Dan Scolnic (U. of Chicago), Dr. Christina Hedges (NASA Ames),
    Nicholas Saunders (U. of Washington),}\\
    {\small Jos\'e Vin\'icius de Miranda Cardoso (Universidade Federal de Campina Grande)}
    \vspace{.5cm}


    \textbf{Google Summer of Code -- The AstroPy Project}\\
    Point spread function photometry for fitting overlapping stars simultaneously
    \hfill \textit{Summer 2016}
    \vspace{.5cm}

    \textbf{National Institute of Science and Technology, USA}\\
    Parameter estimation for photoactivated localization microscopy
    \hfill \textit{Summer 2015}
    \vspace{.5cm}

    \textbf{Institute of Advanced Studies in Communications, Brazil}\\
    Statistical characterization of free space optical channels \hfill \emph{2016 -- 2016}  \\
    Signal detection in generalized fading channels \hfill \emph{2015 -- 2016} \\
    Multiplatform software for objective stereoscopic image and video quality assessment \hfill \emph{2013 -- 2014} \\
    Stereoscopic video quality estimation using objective algorithms \hfill \emph{2012 -- 2013} \\
    Development of a novel objective algorithm for video quality assessment \hfill \emph{2011 -- 2012}

\section*{Publications}
Please, refer to \url{https://mirca.github.io/publications}

\section*{Competencies}
\begin{description}
    \item[Software:] Python (numpy, scipy, pandas, scikit-learn), git/GitHub, C/C++, Unix shell
    \item[Courses:] Stochastic Processes, Information Theory, Random Signal Theory, Estimation and Detection Theory
    \item[Languages:] Native Portuguese, Fluent English
\end{description}

\section*{Awards}
\begin{enumerate}
  \item Selected, with full travel funding, to the workshop \textit{Preparing for TESS}, New York City, USA, 2018
  \item Selected to the workshop \textit{Python in Astronomy}, Leiden, The Netherlands, 2017
  \item Selected, with full travel funding, to the S\~ao Paulo School of Advanced Science on Nanophotonics, S\~ao Paulo, Brazil, 2016
  \item Travel Grant Recipient, IEEE Antennas and Propagation Symposium, Puerto Rico, 2016
  \item Young Author Recognition Award, International Telecommunication Union, ITU Kaleidoscope 2015
  \item Young Author Recognition Award, International Telecommunication Union, ITU Kaleidoscope 2014
  \item The paper ``SQUALES: A QT-based Application for Full-Reference Objective Stereoscopic
      Video Quality Measurement'' was one of the six papers nominated for Best Paper Award at ITU Kaleidoscope 2014
\end{enumerate}

\section*{Additional Information}
\begin{itemize}
    \item[--] Member of the AstroPy software development community.
    \item[--] Participated in the IEEEXtreme 24-Hours Programming Competition in 2013, 2014, 2015, and 2016.
    \item[--] Student of the week on the IEEE Students Facebook webpage.
    \item[--] Participated at the \textit{PSF Photometry and Software Workshop}, Space Telescope Science Institute, Baltimore, 2017.
    \item[--] Attended NASA Ames Machine Learning Workshop, 2017.
\end{itemize}

\end{titlepage}

\end{document}
