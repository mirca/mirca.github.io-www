\documentclass[10pt]{article}
%\usepackage[cmex10]{amsmath}
\usepackage{array}
\usepackage{mdwmath}
\usepackage{wrapfig}
%\usepackage{ae}
\usepackage[T1]{fontenc}
\usepackage{tgpagella}
\usepackage{float}
\usepackage{enumitem}
%\usepackage[latin1]{inputenc}
\usepackage{mdwtab}
\usepackage{amssymb}
\usepackage{eqparbox}
\usepackage{eulervm}
\usepackage{geometry}
\usepackage{subfig}
\usepackage{graphicx}
\usepackage{setspace}
\usepackage{url}
%\usepackage{lmodern}
\geometry{lmargin=2cm,tmargin=2cm,rmargin=2cm,bmargin=1cm}
%opening
\setlength\parindent{0pt}

\begin{document}
\pagestyle{empty}
\begin{titlepage}
     {\Large{\textbf{Jos\'e (Z\'e) Vin\'icius de Miranda Cardoso}}}
     \vspace{.5cm}

    \begin{minipage}[b]{8cm}
     Software Engineer\\
    \end{minipage}
    \hfill
    \begin{minipage}[b]{4cm}
        \hfill \texttt{jvmirca@gmail.com}\\
        \texttt{http://mirca.github.io}\\
        \texttt{GitHub: @mirca}
    \end{minipage}


\section*{Education}

\emph{B.Eng. in Electrical Engineering} \hfill \textit{2019} \\
\textbf{Federal University of Campina Grande}, Brazil\\
Advisor: Dr. Marcelo Sampaio de Alencar
\vspace{.5cm}

\emph{Nanodegree in Machine Learning Engineering} \hfill \textit{2018}\\
\emph{Nanodegree in Artificial Intelligence}\\
\textbf{Udacity}
\vspace{.5cm}

\emph{Visiting Student -- Electrical Engineering and Computer Science} \hfill \textit{Fall 2014 -- Spring 2015} \\
\textbf{The Catholic University of America}, USA\\
\textbf{University of Maryland at College Park}, USA \\
Brazil Scientific Mobility Program, Fully funded scholarship recipient \\
Advisors: Dr. Duilia F. de Mello and Dr. Jandro L. Abot
\vspace{.5cm}

\emph{Technical Degree in Informatics} \hfill \textit{2010}\\
\textbf{Federal Institute of Education, Science and Technology of Para\'iba}, Brazil \\
Advisor: Dr. Carlos Danilo Miranda Regis

\section*{Professional Experience}
\emph{Machine Learning Mentor} \hfill \textit{May 2019  --}
\\\textbf{Udacity}, Remote
\vspace{.5cm}

\emph{Scientific Software Engineering Intern} \hfill \textit{Mar 2017 -- Feb 2018}
\\\textbf{NASA Ames Research Center}, Silicon Valley, USA
\\Kepler/K2 Guest Observer Office
\\Mentor: Dr. Geert Barentsen
\vspace{.5cm}

\emph{Google Summer of Code Student} \hfill \textit{Summer 2016}
\\\textbf{The AstroPy Project}
\\ Project title: Point spread function photometry for fitting overlapping stars simultaneously
    \\Mentors: Dr. Erik Tollerud, Dr. Hans Moritz G\"unther, and Dr. Brigitta Sip\H{o}cz
\vspace{.5cm}

\emph{Undergraduate Teaching Assistant} \hfill \textit{Spring 2015}
\\\emph{Probability and Statistics for Electrical Engineering and Computer Science}
\\\textbf{Federal University of Campina Grande}, Brazil
\vspace{.5cm}

\emph{Undergraduate Research Assistant} \hfill \textit{Fall 2015 -- Fall 2016}
\\\textbf{Institute for Advanced Studies in Communications}, Brazil
\\Mentor: Dr. Marcelo Sampaio Alencar
\vspace{.5cm}

\emph{Undergraduate Guest Researcher} \hfill \textit{Summer 2015}
\\\textbf{National Institute of Standards and Technology}, USA
\\Center for Nanoscale Science and Technology
\\Nanofabrication Research Group
\\Mentor: Dr. Marcelo Ishihara Davan\c co
\vspace{.5cm}

\section*{Volunteering Experience}
\emph{Google Summer of Code Mentor Organization Administrator} \hfill \textit{Summer 2019}
\\Admin for the OpenAstronomy organization during GSoC 2019
\vspace{.5cm}

\emph{Google Summer of Code Mentor for the AstroPy Project} \hfill \textit{Summer 2018}
\\Project title: Develop astropy tutorials on how to fit data
\vspace{.5cm}

\emph{Reviewer for the Brazilian Conference on Signal Processing and Telecommunications} \hfill \textit{Summer 2018}

\section*{Projects}
    \textbf{NASA Transiting Exoplanet Survey Satellite (TESS) Proposal}
    \hfill \textit{2018}\\
    Performing The Most Comprehensive Exoplanet Survey Of The Southern Sky With TESS Full Frame Images\\
    {\small Principal Investigator: Dr. Benjamin Montet (University of Chicago)}\\
    {\small Co-Investigators: Dr. Dan Foreman-Mackey (Flatiron), Dr. Jessie Christiansen (IPAC/Caltech),
    Dr. Rodrigo Luger (U. of Washington), Dr. Dan Scolnic (U. of Chicago), and Dr. Christina Hedges (NASA Ames)}\\
    {\small Undergraduate students: Jos\'e Vin\'icius de Miranda Cardoso (Universidade Federal de Campina Grande) and
    Nicholas Saunders (U. of Washington)}
    \vspace{.5cm}

    \textbf{National Institute of Science and Technology, USA}\\
    Parameter estimation for photoactivated localization microscopy\\
    {\small Mentor: Dr. Marcelo Davanco}
    \hfill \textit{Summer 2015}
    \vspace{.5cm}

\section*{Selected Publications}
\begin{enumerate}
\item \textbf{Cardoso, J. V. M.}, \textit{et. al.} On the Performance of the Energy Detector Subject to Impulsive
Noise in $\kappa - \mu$, $\alpha - \mu$, and $\eta - \mu$ Fading Channels. \textit{Tsinghua Science and Technology}, 2017.
\item Davanco, M., I., Liu, J., Sapienza, L., Zhang, C. Z., \textbf{Cardoso, J., V., M.}, Verma, V., Mirin, R., Nam,
S. W, Srinivasan, K. Heterogeneous integration for on-chip quantum photonic circuits with single quantum dot devices.
\textit{Nature Communications}, 2017.
\item \textbf{Cardoso, J. V. M.}, \textit{et. al.} An approximate exponentiated Weibull envelope-phase distribution.
\textit{IEEE International Symposium on Antennas and Propagation/USNC-URSI National Radio Science Meeting}, Farjado, Puerto Rico, 2016.
\textit{Travel grant recipient}.
\end{enumerate}

For a complete list of my publications, please refer to \url{https://mirca.github.io/publications}.

\section*{Awards}
\begin{enumerate}
  \item Selected, with full travel funding, to the workshop \textit{Preparing for TESS}, New York City, USA, 2018
  \item Selected to the workshop \textit{Python in Astronomy}, Leiden, The Netherlands, 2017
  \item Selected, with full travel funding, to the S\~ao Paulo School of Advanced Science on Nanophotonics, S\~ao Paulo, Brazil, 2016
  \item Travel Grant Recipient, IEEE Antennas and Propagation Symposium, Puerto Rico, 2016
  \item Young Author Recognition Award, International Telecommunication Union, ITU Kaleidoscope 2015
  \item Young Author Recognition Award, International Telecommunication Union, ITU Kaleidoscope 2014
  \item The paper ``SQUALES: A QT-based Application for Full-Reference Objective Stereoscopic
      Video Quality Measurement'' was one of the six papers nominated for Best Paper Award at ITU Kaleidoscope 2014
\end{enumerate}

\section*{Competencies}
\begin{description}
    \item[Coding:] Python (numpy, scipy, pandas, scikit-learn), R, git/GitHub, TensorFlow, C/C++, Unix shell, MATLAB
    \item[Courses:] Stochastic Processes, Information Theory, Random Signal Theory, Estimation and Detection Theory
    \item[Languages:] Native Portuguese, Fluent English
\end{description}

\section*{Additional Information}
\begin{itemize}
    \item[--] Member of the AstroPy software development community
    \item[--] Participated in the IEEEXtreme 24-Hours Programming Competition in 2013, 2014, 2015, and 2016
    \item[--] Student of the week on the IEEE Students Facebook webpage
    \item[--] Participated at the \textit{PSF Photometry and Software Workshop}, Space Telescope Science Institute, Baltimore, 2017
    \item[--] Attended NASA Ames Machine Learning Workshop, 2017
\end{itemize}

\end{titlepage}

\end{document}
